\documentclass{article}%
\usepackage[T1]{fontenc}%
\usepackage[utf8]{inputenc}%
\usepackage{lmodern}%
\usepackage{textcomp}%
\usepackage{lastpage}%
\usepackage{geometry}%
\geometry{margin=2cm,includeheadfoot=True}%
\usepackage{graphicx}%
\usepackage{longtable}%
\usepackage{multirow}%
%
\title{Glees {-} Das beste Bouldergebiet der Welt}%
%
\begin{document}%
\normalsize%
\maketitle%
\section{Ein Bild vom Veitskopf}%
\label{sec:EinBildvomVeitskopf}%


\begin{figure}[h!]%
\centering%
\includegraphics[width=480px]{glees.jpg}%
\caption{Super coole Bildbeschreibung}%
\end{figure}

%
\section{Ein bisschen Text}%
\label{sec:EinbisschenText}%
\subsection{Zu 100\% von der Website runterkopiert}%
\label{subsec:Zu100vonderWebsiterunterkopiert}%
Im Bouldergebiet Glees auf dem Gebiet der Gemeinde Wassenach in der Osteifel wird seit den 80er Jahren geklettert. Das Gebiet zieht sich in einem Waldstück entlang eines Hangs über 2km hin.\newline%
Das Basanit genannte Gestein entstammt einem Ausbruch des Vulkans Veitskopf. Es ist dank Einschlüssen kleiner Quarzkristalle sehr rau und dadurch enorm hautintensiv. Die Kletterei ist geprägt von Leisten an Säulen, die – mal näher, mal weiter – nebeneinanderstehend die Abbruchkante des Lavastroms bilden. Es gibt mittlerweile über 300 erschlossene Linien.

%


\begin{figure}[h!]%
\centering%
\includegraphics[width=240px]{Klettern in Glees.jpg}%
\caption{Klettern in Glees macht Spaß}%
\end{figure}

%
\section{Grade}%
\label{sec:Grade}%
Alle Boulder sind mit einem Pfeil in entsprechender Farbe markiert.%
\begin{longtable}{c c}%
\hline%
Pfeilfarbe&fb{-}Grad\\%
\hline%
\endhead%
\hline%
\multicolumn{2}{c}{}\\%
\hline%
\endfoot%
\hline%
\multicolumn{2}{c}{}\\%
\hline%
\endlastfoot%
gelb&3a {-} 4b\\%
blau&4b {-} 5b\\%
rot&5c {-} 6b\\%
weiß&6b+ {-} 8a\\%
\end{longtable}

%
\section{Eine Tabelle}%
\label{sec:EineTabelle}%
\subsection{Sektoren und Bereiche}%
\label{subsec:SektorenundBereiche}%
\begin{tabular}{|c|c|}%
\hline%
Sektoren&Bereiche\\%
\hline%
\multirow{2}{*}{Bleausard}&Osterinsel\\%
\cline{2%
-%
2}%
&Bleausardblock\\%
\hline%
\multirow{2}{*}{Mauerlay}&Balrock\\%
\cline{2%
-%
2}%
&Rübennase\\%
\cline{2%
-%
2}%
&Megalopolis\\%
\cline{2%
-%
2}%
&Rush More\\%
\cline{2%
-%
2}%
&Supra Arbor\\%
\cline{2%
-%
2}%
&Kamel\\%
\cline{2%
-%
2}%
&Bescherung\\%
\cline{2%
-%
2}%
&Mauerlay\\%
\hline%
\multirow{2}{*}{Kruzifix}&Elefant\\%
\cline{2%
-%
2}%
&Kreuzfels\\%
\cline{2%
-%
2}%
&Eisheiligen\\%
\cline{2%
-%
2}%
&Tiefflieger\\%
\cline{2%
-%
2}%
&Löwenherz\\%
\hline%
\multirow{2}{*}{Lichtung}&Bromarma\\%
\cline{2%
-%
2}%
&Krokodil\\%
\cline{2%
-%
2}%
&Lawine\\%
\cline{2%
-%
2}%
&Lichtung\\%
\hline%
\multirow{2}{*}{Romani Ite Domum}&The Egg\\%
\cline{2%
-%
2}%
&Forum\\%
\cline{2%
-%
2}%
&Römerturm\\%
\cline{2%
-%
2}%
&Es\\%
\cline{2%
-%
2}%
&Windei\\%
\hline%
\multirow{2}{*}{Labyrinth}&Die Macht\\%
\cline{2%
-%
2}%
&Steinzeit\\%
\cline{2%
-%
2}%
&Here Be Dragons\\%
\hline%
\multirow{2}{*}{Terra Incognita}&Voyger\\%
\cline{2%
-%
2}%
&Humboldt\\%
\cline{2%
-%
2}%
&Newton\\%
\hline%
\multirow{2}{*}{Mordor}&Kamikaze\\%
\cline{2%
-%
2}%
&Camelot\\%
\cline{2%
-%
2}%
&Almost Satan\\%
\cline{2%
-%
2}%
&Cosmos\\%
\cline{2%
-%
2}%
&Orodruin\\%
\cline{2%
-%
2}%
&Barad Dur\\%
\cline{2%
-%
2}%
&Gondor\\%
\hline%
\multirow{2}{*}{West{-} und Südblöcke}&Orthank\\%
\cline{2%
-%
2}%
&Macchu Picchu\\%
\cline{2%
-%
2}%
&Südblöcke\\%
\hline%
\end{tabular}

%
\end{document}